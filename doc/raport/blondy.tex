W projekcie zajmowałem się layouterem diagramów. Zarówno wyborem algorytmów jak i ich implementacją oraz projektowaniem struktur wykorzystywanych przy ich realizacji.
Na początku projektu wraz z kolegami opracowaliśmy interfejs modułu realizującego zadania layoutera. Następnie rozpocząłem poszukiwanie algorytmów zapewniających tę funkjconalność. W ich wyniku stworzyłem opracowanie algorytmów, które nadają się do naszych zastosowań. Były to głównie algorytmy grawitacyjne lub sprężynowe. Następnie wspólnie z kolegami ustaliliśmy kolejność w jakiej będziemy chcięli implementować wybrane algorytmy.
W pierwszej kolejności opracowałem, zaimplementowałem i zoptymalizowałem algorytm rozkładu kołowego. Działa rozkładając wierzchołki diagramu na okręgu minimalizując liczbę przecięć krawędzi poprzez umieszczanie sąsiadujących wierzchołków blisko siebie. Algorytm ten daje doskonałe wyniki w przypadku grafów gęstych o niewielkiej liczbie wierzchołków.
Po zakończeniu fazy testów przeszedłem do implementacji pierwszego z algorytmów wytypowanych do umieszczenia w naszym projekcie - algorytmu Eadesa. Pseudokod algorytmu był bardzo ogólny i wymagał w dużej mierze własnej interpretacji. Po zaimplementowaniu go okazało się, że nie działa poprawnie. Po wielu nieudanych próbach poprawienia efektów jego działania zdecydowaliśmy się skorzystać z gotowej implementacji z biblioteki Graphviz.
Niestety spędziłem tak wiele czasu na próbach poprawienia wyników algorytmu że nie wystarczyło mi czasu na implementację żadnego innego algorytmu z naszej listy.
Planowałem również zaimplementować algorytm grupujący niektóre węzły, których sposób rozłożenia jest z góry znany (np. generalizacja). Algorytm rozkładania miał operować zarówno na pojedynczych wierzchołkach jak i na ich grupach. Niestety tego pomysłu również nie udało się zrealizować ze względu na brak czasu.
