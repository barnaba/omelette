W projekcie zajmowałem się projektowaniem oraz implementacją warstwy aplikacji służącej do rysowania diagramów. Analizowałem oraz podejmowałem decyzje na temat komunikacji modułów kompilatora, rysowania oraz GUI. Przez cały czas życia projektu utrzymywałem moduł tworzenia diagramów z linii komend który wymagał częstej refaktoryzacji ze względu na dynamikę zmian w innych modułach, z których korzysta.
Podczas projektu mocno podniosłem swoją wiedzę z zakresu Pythona oraz biblioteki Qt. Cały czas szukałem nowych rozwiązań dla istniejących (oraz tych które miały dopiero się ukazać) problemów. Jedyną niepokonaną przeszkodą okazała się generacja diagramów z pozimu konsoli w środowisku bez X-Server. Po wielu dniach poszukiwań oraz nieprzespanych nocy spędzonych na debugowaniu, doszedłem do wnisku, że jedyną możliwością jest użycie zewnętrznej biblioteki do rysowania czcionek - Qt polega na X-Serverze w sprawach czcionek. Uznaliśmy, że koszt jest większy niż zysk i zrezygnowaliśmy z możliwości generowania diagramów w środowisku bez ,,X-ów''.