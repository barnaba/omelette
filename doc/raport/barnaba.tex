W początkowej fazie projektu zajmowałem się głównie planowaniem architektury i sposobu działania kompilatora.
Potem zacząłem implementować klasy, które miały leżeć u podstaw tej architektury (takie jak \textbf{UMLObject}).
Na tym etapie projektu znacząca część mojej aktywności polegała na pomaganiu niektórym kolegom w opanowaniu stosowanych technologii (głównie obsługi repozytorium) i czytaniu książki ``Dive into Python''.
Zajmowałem się też zarządzaniem repozytorium --- tworzeniem struktury katalogów, paczek, wybieraniem i sprawdzaniem możliwości frameworków, używanych do testowania.

W dalszych etapach projektu zajmowałem się też modułem common, który zawierał generyczne klasy, na których miały bazować konkretne klasy środowiska graficznego reprezentujące obiekty na diagramie.
Wprowadzałem też drobne zmiany w modułach \texttt{actions}, \texttt{code}, \texttt{parser}, \texttt{layouter}, \texttt{logging}.
Moduły \texttt{common} i \texttt{uml} (a także ich testy) wymagały wielokrotnego poprawiania.
Dzięki temu zyskałem praktykę w pisaniu dobrych (nie specyficznych) testów jednostkowych.
%\href{http://dl.dropbox.com/u/10621643/actually.png}{Mimo tej praktyki, jak ktos cos zmieni w w/w modułach, to będzie musiał przepisać wszystkie testy}
