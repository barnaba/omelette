\documentclass[a4paper,11pt,notitlepage]{article}
\usepackage[utf8]{inputenc}	% latin2 - kodowanie iso-8859-2; cp1250 - kodowanie windows
\usepackage[T1]{fontenc}
\usepackage[polish]{babel}
\usepackage[MeX]{polski}
\selectlanguage{polish}
%\hyphenation{}
%\linespread{1.5}
\usepackage[top=2cm, bottom=2cm, left=2cm, right=2cm]{geometry}

\author{
  Barnaba Turek\and
  Bartosz Pieńkowski\and
  Michał Zochniak \and
  Sławomir Blatkiewicz\and
  Jakub Górniak\and
  Piotr Piechal
}
\title{Projekt zespołowy - wstępne rozpoznanie problemu}
%todo nazwa projektu. Moja propozycja to "tUML" t=text albo "tUMLgen" gen=generator albo "gUMt" = generate UML from text
\date{\today}
\begin{document}
\maketitle %przedefiniować maketitle aby wypisał autorów jeden pod drugim
%todo przedefiniować maketitle i toc aby były osobne strony
\tableofcontents
\section{Przedmiot projektu}
Projekt obejmuje cały proces powstawania oprogramowania umożliwiającego generowanie diagramów UML w postaci graficznej na podstawie plików tekstowych o określonej strukturze opartej na konkretnej składni.
\section{Ograniczenia projektu}
\subsection{Zasoby czasowe}
Projekt ma trwać około 9 miesięcy. Nieprzekraczalny termin zakończenia prac nad projektem to 10 czerwca 2011.
\subsection{Zasoby ludzkie}
Do realizacji projektu przydzielony został zespół 6 programistów w składzie:
  \begin{itemize}
    \item{Barnaba Turek}
    \item{Bartosz Pieńkowski}
    \item{Michał Zochniak}
    \item{Sławomir Blatkiewicz}
    \item{Jakub Górski}
    \item{Piotr Piechal}
  \end{itemize}
\section{Rozpoznanie problemu}
Podstawą całego projektu jest stworzenie języka, który umożliwi precyzyjny opis elementów diagramu klas w notacji UML, zgodnie ze standardami wersji 2.0.
Dodatkowymi modułami potrzebnymi do realizacji tego rozwiązania będą parser oraz generator plików graficznych.
\subsection{Język}
Język powinien umożliwiać opisanie podstawowych elementów diagramu klas, opisanych w specyfikacji notacji UML 2.0:
\begin{itemize}
%todo uzupełnijcie jeśli o czymś zapomniałem
\item{klasa}
\item{stereotyp}
\item{relacja (asocjacja, agregacja, generalizacja) wraz z określeniem liczebności i ról}
\item{notatka}
\end{itemize}
Dodatkowo zakładamy iż język powinien umożliwiać definiowanie:
\begin{itemize} %tutaj też proszę o ewentualne uzupełnienie
\item{relacji n-arnej}
\item{klasy asocjacyjnej}
\end{itemize}
\subsection{Parser}
%todo jak realizujemy parser? nie znam się na tym ale mówiliście o wyrażeniach regularnych itp - chodzi o samą ideę
\subsection{Generator plików graficznych}
Generator plików graficznych powinien domyślnie korzystać z formatu JPEG. Dodatkowo powinien automatycznie optymalizować ułożenie elementów na diagramie. %można też napisać ogólnie o idei tej optymalizacji, choć pewnie trzebaby o tym trochę poczytać.
\section{Założenia projektu}
\subsection{Cel podstawowy}
Celem podstawowym jest stworzenie programu sterowanego z linii komend, który wygeneruje plik graficzny zawierający diagram klas odwzorowujący wskazany plik tekstowy zawierający kod w utworzonym języku.
\subsection{Cele dodatkowe}
Celami dodatkowymi, których realizacja rozważona zostanie po osiągnięciu celu podstawowego są:
\begin{enumerate}
  \item{
    utworzenie zintegrowanego środowiska programistycznego (IDE) do tego języka, w skład którego wchodziłyby następujace elementy:
    \begin{itemize}
      \item{edytor tekstowy oferujący kolorowanie składni, oraz inteligentne formatowanie tekstu}
      \item{podgląd diagramu na bieżąco}
    \end{itemize}
  }
  \item{rozszerzenie funkcjonalności IDE o możliwość redefiniowania położenia poszczególnych elementów na diagramie w trybie graficznym (\emph{drag and drop}).}
  \item{rozszerzenie funkcjonalności generatora plików graficznych o możliwość wyboru formatu, w jakim ma zostać zapisany obraz spośród następujących:
    \begin{itemize}
      \item{GIF}
      \item{PNG}
    \end{itemize}
  } %wiem że o tym nie rozmawialiśmy, ale myślę, że taki feature może być przydatny np żeby nie używać dodatkowych programów graficznych do zmniejszania lub konwersji plików
\end{enumerate}
\section{Proponowane rozwiązania}
%todo napisać przykład syntaxu
\section{Proponowane technologie}
Proponujemy do osiągnięcia celu głównego wykorzystanie technologii języka Python. Za takim rozwiązaniem przemawiają następujące argumenty:
\begin{enumerate}
  \item{przenośność rozwiązania spowodowana skryptowym charakterem języka}
  \item{łatwość użytkowania - brak potrzeby instalacji oprogramowania do jego poprawnego działania}
  \item{aspekt dydaktyczny - chęć zapoznania się z proponowaną technologią}
\end{enumerate}
Do realizacji generatora plików graficznych proponujemy użycie biblioteki Cute.
%todo ponieważ?
%todo czego użyjemy do IDE?
\end{document}
